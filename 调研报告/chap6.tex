\chapter{我们要做什么}

由于 Unikernel 的结构组成的特殊性,一些程序想要移植到 Unikernel 上往往需要将代码完全重写(而且现在比较成熟的Unikernel实现往
往需要使用冷门的编程语言,开发接口的难度也很大),这一定程度影响了可移植性。而且 Unikernel 并不是传统的程序,开发和测试都与传统
的有不小的差别,造成了开发难度的增大。

在开发过程中,开发者可以在传统的操作系统上进行开发,而所有内核相关的功能,暂且由开发机的操作系统提供。

而在测试环境中,大部分 Unikernel 的实现会将应用代码与需要的内核模块构建成 Unikernel 后,再将其跑在一个传统的操作系统上,利用
传统操作系统上的工具来测试 Unikernel。以 Rumprun为例,它可以通过 KVM / QEMU 来运行一个 Rumprun Unikernel VM,随后用 Ho
st OS 上的 GDB 来对其进行调试,这跟我们实验中测试内核的步骤比较相似,那么我们在实验中对于内核调试的种种遗憾,如调试器容易有 bug
,编译调试步骤繁琐,命令或者说功能不够丰富科学等等,也可以投影到 Unikernel 的开发调试上。联想诸如 Pycharm ,Vs Studio 这些成
熟的 IDE,它们丰富的调试辅助功能,对开发效率的提高是非常显著的,也变相地降低了编程开发测试的难度,降低了编程的门槛,促进了语言的普
及。那么,如果我们对于 Unikernel 的测试环境能有所改进,小到改进 debugger 一些缺点,大到开发一个相应的 IDE,就可能为促进 Unik
ernel 的实用化和普及,创建一个更高效的云计算世界做出一份贡献。这正是我们想要做的。