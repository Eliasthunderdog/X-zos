\chapter{Unikernel的改进可能}

\section{更进一步:Unikernel Monitor}

现在Unikernel的结构是这样的

也就是说,我们用Unikernel的方法Specialize的部分是OS,然而,其实在Unikernel下面一层的Monitor也可以进行Specialize,这样它
们直接运行在hypervisor之上(比如xen和kvm),这样做的好处当然是进一步优化了软件栈,可以获得更高的运行效率。

\section{Debug环境的优化}

Unikernel的程序运行方式决定它很难像普通的应用程序一样进行调试,但是这并不代表Unikernel是完全不可调试的,其实我们已经不能像看待
传统的虚拟机一样看待Unikernel,应该对Unikernel运行过程中出现的意外情况有相应的保护措施和追踪记录,以此方便开发者进行维护。

\section{Unikernel和Docker结合}

Unikernel和Docker的思想类似--打包,但是二者各有优缺点和合适的使用场景,二者不是竞争关系,并不存在一个淘汰另一个之说。

2015年的Hyper项目实现了在hypervisor上面运行一个Docker,相当于拿掉了操作系统一层,这样的技术就与Unikernel非常类似。这么做大
大限制了攻击者的空间,提高了Docker的安全性。